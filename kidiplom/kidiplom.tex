\documentclass[
%  master,
%  program=infpvs,
%  printversion,
  biblatex,
%  language=english,
%  font=sans,
  figures=false,
  glossaries,
  index
]{kidiplom}

\title{Algoritmy pro problém obchodního cestujícího}
\title[english]{Algorithms for the travelling salesman problem}

\author{Kateřina Sáňková}

\supervisor{Mgr. Petr Osička, Ph.D.}

\yearofsubmit{\the\year}

\annotation{Anotace - jeden odstavec}

\annotation[english]{Anotace anglicky}

\keywords{problém obchodního cestujícího;}
\keywords[english]{travelling salesman problem;}

\thanks{Děkuji, děkuji, děkuji.}

%% Cesta k souboru s bibliografií pro její sazbu pomocí BibLaTeXu
%% (zvolenou nepovinným parametrem biblatex makra
%% \documentclass). Použijte pouze při této sazbě, ne při (výchozí)
%% sazbě v prostředí thebibliography.
\bibliography{bibliografie.bib}

%% Další dodatečné styly (balíky) potřebné pro sazbu vlastního textu
%% práce.
\usepackage{lipsum}
\usepackage{longtable}

\begin{document}
%% Sazba úvodních stran -- titulní, s bibliografickými údaji, s
%% anotací a klíčovými slovy, s poděkováním a prohlášením, s obsahem a
%% se seznamy obrázků, tabulek, vět a zdrojových kódů (pokud jejich
%% sazba není vypnutá).
\maketitle

%% Vlastní text závěrečné práce. Pro povinné závěry, před přílohami,
%% použijte prostředí kiconclusions. Povinná je i příloha s obsahem
%% přiloženého datového média.

%% -------------------------------------------------------------------

\newcommand{\BibLaTeX}{\textsc{Bib}\LaTeX}

\section{Teorie}
Problém obchodního cestujícího je úzce spjatý s \textit{teorií grafů} a proto je potřeba si na úvod zavést některé z jejich základních pojmů.

\subsection{Graf}
Graf je jedna ze základních reprezenrtací prvků množiny objektů a jejich vzájemných propojení. Takovým objektům budeme říkat \textit{vrcholy} (někdy také \textit{uzly}) a propojením \textit{hrany}. Uvažujeme-li orintaci hran, pak říkáme, že je graf \textit{orientovaný}, jinak \textit{neorientovaný}.

\begin{definition}[Neorientovaný graf]
\textit{Neorientovaný graf} je dvojice $\langle V, E \rangle$, kde $V$ je neprázdná množina vrcholů a $E \subseteq \{\{u, v\} \mid u, v \in V$, $u \neq v \}$ je množina (\textit{neorientovaných}) hran.
\end{definition}

\begin{definition}[Orientovaný graf]
\textit{Orientovaný graf} je dvojice $\langle V, E \rangle$, kde $V$ je neprázdná množina vrcholů a $E \subseteq \{ \langle u, v \rangle \mid u, v \in V\}$ je množina (\textit{orientovaných}) hran.
\end{definition}

$u, v$ nazýváme \textit{koncové uzly} hrany.

\subsubsection{Stupeň vrcholu} 
V některých situacích nás bude zajímat počet hran, kterým je jistý uzel koncovým. Tomuto číslu budeme říkat \textit{stupeň vrcholu} $u$ a budeme ho značit $deg(u)$. Důležité bude také následující tvrzení.

\begin{theorem}
V každém grafu $G=\langle V, E \rangle$ platí, že $\sum_{v \in V} deg(v) = 2 |E|$.
\end{theorem}

\subsubsection{Hranové ohodnocení}
U problému obchodního cestujícího chceme, aby hrany vstupních grafů měly nějakou váhu. Tu jim přiřazuje tzv. \textit{hranové ohodnocení} definované následnovně:
$$w : E \rightarrow D$$
kde $w$ je \textit{hranové ohodnocení}, $E$ je množina hran příslušného grafu a $D$ je nějaká množina hodnot.

\subsubsection{Úplný graf}
V knihovně se také bude počítat s tím, že na vstupu bude \textit{úplný} graf.******************
\begin{definition}
Neorientovaný graf nazýváme \textit{úplný}, pokud platí, že každé dva jeho vrcholy jsou spojeny hranou.
\end{definition}

\subsubsection{Kostra grafu}
Některé z algoritmů v knihovně jsou založené na hledání tzv. \textit{minimální kostry grafu} . Před zavedením tohoto pojmu je ještě nutné si definovat, co je to \textit{souvislý graf} a \textit{podgraf} grafu.

\begin{definition}[Souvislý graf]
Neorientovaný graf $G=\langle V, E \rangle$ nazýváme \textit{souvislý}, pokud $\forall u,v \in V$ existuje sled (viz později) z $u$ do $v$.
\end{definition}

\begin{definition}[Podgraf]
Graf $G_2=\langle V_2, E_2 \rangle$ nazýváme \textit{podgraf} grafu $G=\langle V, E \rangle$, právě když $V_2 \subseteq V$ a $E_2 \subseteq  E$.
\end{definition}

\begin{definition}[Kostra grafu]
Kostra neorientovaného grafu je jeho souvislý podgraf, který obsahuje všechny jeho vrcholy a nevyskytují se v něm žádné kružnice.
\end{definition}

Pokud mají hrany původního grafu přiřazené váhy příslušným hranovým ohodnocením $w$, potom můžeme uvažovat o \textit{minimální kostře grafu}. Tou budeme rozumnět právě takovou kostru $MSP = \langle V, E' \rangle$, která bude mít mezi ostatními minimální součet  $\sum_{e \in E'} w(e)$.

\subsection{Cestování v grafech}

\begin{definition}
\textit{Sledem} v grafu $G=\langle V, E \rangle$ rozumíme posloupnost $v_0, e_1, v_1, e_2, v_2, \cdots, e_n, v_n$, kde $\forall i \in \{0,\cdots, n\} \ v_i \in V$ a $\forall j \in \{1, \cdots, n\} \ e_j \in E$.
\end{definition}



\pagebreak

\section{Pojmy}
Toto jsou styly pro psaní bakalářských a diplomových prací přes typografický systém \LaTeX{}, tedy \textbf{kistyles}.

\subsection{Požadavky a podprovaná prostředí}
Sada balíku \textbf{kistyles} podporuje následující distribuce systému \LaTeX{}:
\begin{itemize}
\item \TeX{} Live.
\end{itemize}



Jsou podporovány všechny výstupní ovladače, tedy jak \textbf{dvi}, tak \textbf{pdf} i \textbf{ps}. Funkčnost zmiňovaných distribucí byla ověřena na několika operačních systémech, mezi které patří:
\begin{enumerate}
\item Windows $8.1$,
\item Archlinux,
\item Debian GNU/Linux.
\end{enumerate}

Důrazně se doporučuje používat aktuální verzi dané distribuce systému \LaTeX{}.

%%%  Po přeložení programem CSLaTeX (třikrát) je potřeba použít
%%%  program DVIPS a takto získaný PostScriptový soubor vytisknout
%%%  na PostScriptové tiskárně nebo pomocí programu GhostScript.
%%%
%%%  Rovněž je možné použít program DVIPDFM a vytvořit z dokumentu
%%%  soubor ve formátu PDF včetně hypertextových odkazů.

\subsection{Přepínače}
Styl kidiplom je z hlediska uživatele zastoupen ekvivalentně nazvanou třídou, kterou je třeba volat na záčátku dokumentu:
\begin{kicode}{TeX}{}{Volání třídy \textbf{kidiplom}}
\documentclass[
  master=true,
  font=sans,
  printversion=false,
  joinlists=true,
  glossaries=true,
  figures=true,
  tables=true,
  sourcecodes=true,
  theorems=true,
  bibencoding=utf8,
  language=czech,
  encoding=utf8,
  program=infoi,
  index=true,
  biblatex=true
]{kidiplom}
\end{kicode}

Následuje přehled přepínačů, je vždy uvedeno jméno přepínač, včetně výchozí hodnoty. Přepínače uvádí tabulka \ref{tab:prepinace}.

%\begin{table}
\begin{center}
\begin{longtable}{>{\bfseries}l >{\ttfamily}c L{8cm}}
\caption{Seznam přepínačů}\label{tab:prepinace}\\
  {\normalfont Přepínač} & {\normalfont Výchozí hodnota} & {\normalfont Popis} \\
\hline
master & false & Povolí nebo zakáže režim diplomové práce. Výchozí režim je tedy bakalářská práce. \\

program & \vtop{\hbox{\strut infpvs}\hbox{\strut ainfvs}} & Specifikuje studijní program/obor (specializaci):\newline
\begin{description}
\item[infoi] Informatika (Obecná informatika)\,--\,bakalářský i navazující magisterský,
\item[infpvs] Informatika (Programování a vývoj software)\,--\,bakalářský,
\item[itp] Informační technologie\,--\,bakalářský, prezenční forma,
\item[itk] Informační technologie\,--\,bakalářský, kombinovaná forma,
\item[infui] Informatika (Umělá inteligence)\,--\,navazující magisterský,
\item[ainfvs] Aplikovaná informatika (Vývoj software)\,--\,navazující magisterský,
\item[ainfpst] Aplikovaná informatika (Počítačové systémy a technologie)\,--\,navazující magisterský,
\item[infv] Informatika pro vzdělávání\,--\,bakalářský,
\item[uinf] Učitelství informatiky pro střední školy\,--\,navazující magisterský,
\item[binf] Bioinformatika\,--\,bakalářský i navazující magisterský,
\item[inf] Informatika (bez specializací)\,--\,bakalářský i navazující magisterský,
\item[ainfp] Aplikovaná informatika (bez specializací)\,--\,bakalářský, prezenční forma,
\item[ainfk] Aplikovaná informatika (bez specializací)\,--\,bakalářský, kombinovaná forma,
\item[ainf] Aplikovaná informatika (bez specializací)\,--\,navazující magisterský.
\end{description} \\

font & serif & Zapne či vypne podporu pěkného bezpatkového fontu. Možné hodnoty jsou:\newline
\begin{description}
\item[sans] Bezpatkové písmo (písmo Iwona).
\item[serif] Patkové písmo (písmo Computer Modern).
\end{description} \\

%%  'encoding=kódování' pro kódování tohoto a vložených zdrojových
%%  textů v kódování jiném než výchozím utf8
encoding & utf8 & Kódování souboru dokumentu, doporučuje se ponechat výchozí hodnotu. \\

bibencoding & utf8 & Kódování souboru bibliografie. Tato volba má smysl pouze, pokud je použita bibliografie skrze balíček \BibLaTeX{}. \\

language & czech & Jazyk práce. \\

printversion & false & Je-li zapnuto, pak budou odkazy vysázeny optimalizovaně pro knižní sazbu. Tuto volbu je nutno použít pro tisk práce. \\

%%% Nepovinné argumenty `tables' a `figures' použijte pouze v případě,
%%% že váš dokument obsahuje tabulky a obrázky a chcete vytvořit
%%% jejich seznamy za obsahem.
%%%
%%% Argument `joinlists' způsobí zřetězení obsahu a seznamů tabulek a obrázků.
%%% Není-li použít, všechny seznamy jsou uvedeny na samostatných stránkách.

joinlists & true & Je-li zapnuto, pak seznamy obrázků, tabulek, vět a
zdrojových kódů sázené za obsahem nebudou rozděleny na samostatné stránky. \\

figures & true & Je-li zapnuto, pak v seznamech položek bude zahrnut seznam obrázků. \\

tables & true & Je-li zapnuto, pak v seznamech položek bude zahrnut seznam tabulek. \\

theorems & false & Je-li zapnuto, pak v seznamech bude zahrnut seznam teorémů. \\

sourcecodes & false & Je-li zapnuto, pak v seznamech bude zahrnut seznam zdrojových kódů. \\

glossaries & false & Je-li zapnuto, pak na konci dokumentu bude vysázen seznam zkratek. \\

index & false & Zapíná podporu sazby rejstříku. \\

biblatex & true & Zapne sazbu bibliografie přes balík \BibLaTeX{}.
\end{longtable}
\end{center}
%\end{table}

\subsection{Geometrie stránky}
Tento styl používá list velikosti $A4$. Pro sazbu prací je třeba použít jednostrannou sazbu. Levý okraj je rozšířen s ohledem na vazbu výsledné knižní podoby práce.

\section{Sazba částí dokumentu}
\subsection{Sazba úvodní strany či obsahu}
Vysázení všech podstatných částí úvodu práce obstará makro \kiinlinecode{TeX}{!}{\\maketitle}. Pro správné vysázení všech částí a meta-informací je potřeba použí makra \kiinlinecode{TeX}{!}{\\title}, \mbox{\kiinlinecode{TeX}{!}{\\author}} a další. Jejich přehled lze najít ve zdrojovém souboru tohoto dokumentu. V případě použítí \textbf{pdf} výstupu se generuje i dodatečná hlavička souboru s meta-informacemi jako je autor dokumentu, název práce či dalšími.

\subsection{Závěry}
Závěr práce by se měl poskytnout jak v původním jazyce práce, tak v jazyce anglickém. Pro sazbu závěru jsou k dispozici příslušná makra. Berte na vědomí, že v anglickém závěru se aktivuje plně anglická sazba se všemi konvencemi. Tedy je třeba používat anglické uvozovky a další správné typografické prvky.

\begin{kicode}{TeX}{}{Sazba závěrů}
% Tiskne český závěr práce.
\begin{kiconclusions}
Závěr práce v \uv{českém} jazyce.
\end{kiconclusions}

% Tiskne anglický závěr práce.
\begin{kiconclusions}[english]
Thesis conclusions written in \uv{English}.
\end{kiconclusions}
\end{kicode}

\subsection{Matematika}
Pro sazbu matematiky je k dispozici sada standardních maker.
$$\langle f \rangle, \lfloor g \rfloor,
\lceil h \rceil, \ulcorner i \urcorner$$

$$\left\{\frac{x^2}{y^3}\right\}$$

$$
A_{m,n} =
 \begin{pmatrix}
  a_{1,1} & a_{1,2} & \cdots & a_{1,n} \\
  a_{2,1} & a_{2,2} & \cdots & a_{2,n} \\
  \vdots  & \vdots  & \ddots & \vdots  \\
  a_{m,1} & a_{m,2} & \cdots & a_{m,n}
 \end{pmatrix}
$$

$$
M = \begin{bmatrix}
       \frac{5}{6} & \frac{1}{6} & 0           \\[0.3em]
       \frac{5}{6} & 0           & \frac{1}{6} \\[0.3em]
       0           & \frac{5}{6} & \frac{1}{6}
     \end{bmatrix}
$$

\subsection{Sazba literatury}
Pro sazbu literatury má uživatel dvě možnosti. Může použít služeb balíků \BibLaTeX{}, který je pro \textbf{kistyles} zapnutý, či lze použít manuální sazbu bibliografie.
\subsubsection{Sazba bibliografie přes \BibLaTeX{}}
Při použití tohoto balíku se data o použité literatuře ukládají do dedikovaného textového souboru, ukázku najdete i v tomto stylu pod jménem \kiinlinecode{text}{!}{bibliografie.bib}.

Formát daného souboru je nad rámec této dokumentace a je na každém uživateli, aby si jej nastudoval. Bibliografie se tiskne makrem \kiinlinecode{TeX}{!}{\\printbibliography}. Taktéž v preambuli dokumentu je třeba definovat, který soubor data bibliografie obsahuje, tedy například \kiinlinecode{TeX}{!}{\\bibliography\{bibliografie.bib\}}.

Dokument, který využívá \BibLaTeX{} je následně nutné přeložit jak pomocí překladače zvoleného ovladače, tak pomocí aplikace \kiinlinecode{text}{!}{biber}. Více informací poskytne soubor \kiinlinecode{text}{!}{Makefile} z distribuce tohoto stylu.

Výhodou tohoto přístupu je, že bibliografie se vysází automaticky a (obvykle) není třeba manuální úprava formátování.

\subsubsection{Manuální sazba bibliografie}
Manuální sazba obnáší vysázení prostředí \kiinlinecode{text}{!}{thebibliography} ručně. To je nad rámec tohoto dokumentu. Ukázku tohoto přístupu lze samozřejmě nalézt ve zdrojovém souboru tohoto dokumentu nebo také \href{http://www.math.uiuc.edu/~hildebr/tex/bibliographies.html}{zde}.

Pro aktivaci manuální sazby bibliografie je třeba volat třídu \kiinlinecode{text}{!}{kidiplom} s parametrem \kiinlinecode{text}{!}{biblatex=false}. Mějte, prosím, na paměti, že v tomto módu jsou makra \kiinlinecode{text}{!}{\\bibliography} a \kiinlinecode{text}{!}{\\printbibliography} nedostupná.

\subsection{Drobná makra}
Základní styl definuje hned několik maker pro usnadnění práce. Například makro \kiinlinecode{TeX}{!}{\\buno} vysází řetezec \uv{bez újmy na obecnosti}. Je k dispozici i verze s prvním velkým písmenem, \kiinlinecode{TeX}{!}{\\Buno}.

Je rovněž možno přidávat položky do seznamu zkratek. K tomu slouží makro \mbox{\kiinlinecode{TeX}{!}{\\newacronym}}, které lze použít například jednoduše jako \kiinlinecode{TeX}{!}{\\newacronym\{UPOL\}\{UPOL\}\{\\kitextunivcz\}}. Na danou zkratku se pak lze odkazovat jednoduše, \mbox{\kiinlinecode{TeX}{!}{\\gls\{UPOL\}}}.

Sazba uvozovek respektuje nastavení částí dokumentu, a proto se doporučuje používat makro \kiinlinecode{TeX}{!}{\\uv}. V anglické závěru práce toto platí taky, viz tato PDF ukázka.

Styl podporuje sazbu odstavců v tabulkách, více obsahuje tabulka \ref{tab:odstavce}.

\begin{table}
\begin{center}
\caption{Odstavce v tabulkách}\label{tab:odstavce}
\begin{tabular}{L{4cm}|R{4cm}|L{4cm}}
\lipsum[23] & \lipsum[22] & \lipsum[21]
\end{tabular}
\end{center}
\end{table}

K dispozici jsou také makra pro sazbu \csharp{} (\kiinlinecode{TeX}{!}{\\csharp}) či \cpp{} (\kiinlinecode{TeX}{!}{\\cpp}).

%% v případě tvorby rejstříku přeložit vygenerovaný soubor .idx
%% programem Makeindex a v případě tvorby seznamu zkratek spustit
%% program Makeglossaries s parametrem jméno souboru zdrojového textu
%% bez přípony a následně opět (dvakrát) přeložit zdrojový text
%% programem pdfLaTeX.

\subsection{Sazba rejstříku}
Sazba rejstříku sestává z několika kroků:

\begin{enumerate}
\item Je třeba přes volbu \kiinlinecode{TeX}{!}{index=true} rejstříkování povolit.
\item Použítím makra \kiinlinecode{TeX}{!}{\\index} rejstříkovat vybrané pojmy.
\item Kompilovat s použitím utility \kiinlinecode{TeX}{!}{makeindex}. Pro specifika tohoto kroku si stačí prohlédnout soubor \kiinlinecode{text}{!}{Makefile}.
\end{enumerate}

Makro \kiinlinecode{TeX}{!}{\\index} je redefinováno tak, že sází klikací odkaz na výraz v rejstříku. Je doporučeno jej použít ihned za výrazem\index{výraz}.

\textbf{Omezení redefinovaného makra \kiinlinecode{TeX}{!}{\\index}}: klikací odkaz nefunguje, pokud použijete konstrukci \kiinlinecode{TeX}{!}{\\index\{výraz|makro\}} (resp. \kiinlinecode{TeX}{!}{\\index\{výraz|(makro\}}), např. \kiinlinecode{TeX}{!}{\\index\{výraz|textit\}}.

Rejstřík lze vysázet pomocí makra \kiinlinecode{TeX}{!}{\\printindex}.

\subsection{Sazba zdrojových kódů}
Styl nabízí dva způsoby sazby zdrojových kódů:

\begin{enumerate}
\item Sazbu řádkových kódů, například \kiinlinecode{CSS}{!}{background-color: white;}. K tomu slouží makro formátu \kiinlinecode{TeX}{!}{\\kiinlinecode\{jazyk\}\{separátor\}\{kód\}}. Za separátor je vhodné volit jakýkoliv znak, který se nevyskytuje v samotném sázeném zdrojovém kódu. Za jazyk je nutno dosadit jeden z těchto: C, TeX, PHP, HTML, Lisp, SQL, TeX, Python, Java, TutorialD, text, csharp, cpp, JavaScript, CSS.

\item Sazbu zdrojových kódu do separátních prostředí. Takto vytištěný kód se objeví v seznamu zdrojových kódů. Ukázka například zdrojový kód \ref{kod:cpp}. Ukázku sazby naleznete ve zdrojovém kódu tohoto dokumentu.
\end{enumerate}

\newacronym{UPOL}{UPOL}{\kitextunivcz}

\begin{definition}[Název definice]
Abcd. Abcd. Abcd. Abcd. Abcd. Abcd. Abcd. Abcd. Abcd. Abcd. Abcd. Abcd. Abcd. Abcd. Abcd. Abcd. Abcd. Abcd. Abcd. Abcd. Abcd. Abcd. Abcd. Abcd. Abcd. Abcd. Abcd. Abcd. Abcd. Abcd. \gls{UPOL}
\end{definition}

\begin{proof}[Název důkazu]
Abcd. Abcd. Abcd. Abcd. Abcd. Abcd. Abcd. Abcd. Abcd. Abcd. Abcd. Abcd. Abcd. Abcd. Abcd. Abcd. Abcd. Abcd. Abcd. Abcd. Abcd. Abcd. Abcd. Abcd. Abcd. Abcd. Abcd. Abcd. Abcd. Abcd.
\end{proof}

\begin{remark}[Pumpovací věta]
Abcd. Abcd. Abcd. Abcd. Abcd. Abcd. Abcd. Abcd. Abcd. Abcd. Abcd. Abcd. Abcd. Abcd. Abcd. Abcd. Abcd. Abcd. Abcd. Abcd. Abcd. Abcd. Abcd. Abcd. Abcd. Abcd. Abcd. Abcd. Abcd. Abcd.
\end{remark}

\begin{example}[Pumpovací věta]
Abcd. Abcd. Abcd. Abcd. Abcd. Abcd. Abcd. Abcd. Abcd. Abcd. Abcd. Abcd. Abcd. Abcd. Abcd. Abcd. Abcd. Abcd. Abcd. Abcd. Abcd. Abcd. Abcd. Abcd. Abcd. Abcd. Abcd. Abcd. Abcd. Abcd.
\end{example}

\begin{lemma}[Název definice]
Abcd. Abcd. Abcd. Abcd. Abcd. Abcd. Abcd. Abcd. Abcd. Abcd. Abcd. Abcd. Abcd. Abcd. Abcd. Abcd. Abcd. Abcd. Abcd. Abcd. Abcd. Abcd. Abcd. Abcd. Abcd. Abcd. Abcd. Abcd. Abcd. Abcd.
\end{lemma}

\begin{consequence}[Název důkazu]
Abcd. Abcd. Abcd. Abcd. Abcd. Abcd. Abcd. Abcd. Abcd. Abcd. Abcd. Abcd. Abcd. Abcd. Abcd. Abcd. Abcd. Abcd. Abcd. Abcd. Abcd. Abcd. Abcd. Abcd. Abcd. Abcd. Abcd. Abcd. Abcd.
\end{consequence}

\begin{theorem}[Pumpovací věta]
Abcd. Abcd. Abcd. Abcd. Abcd. Abcd. Abcd. Abcd. Abcd. Abcd. Abcd. Abcd. Abcd. Abcd. Abcd. Abcd. Abcd. Abcd. Abcd. Abcd. Abcd. Abcd. Abcd. Abcd. Abcd. Abcd. Abcd. Abcd. Abcd. Abcd.
\end{theorem}


\begin{kicode}{cpp}{kod:cpp}{\cpp}
int main("cs acsa") // komentar
int main("cs acsa") // komentar
int main("cs acsa") // komentar
int main("cs acsa") // komentar
int main("cs acsa") // komentar
\end{kicode}

\begin{kicode}{JavaScript}{}{JS}
new object() // komentar
\end{kicode}

\begin{kicode}{csharp}{}{\csharp}
public static int main("cs acsa") // komentar
\end{kicode}

\begin{kicode}{SQL}{}{SQL}
SELECT * FROM table_1; /* komentar */
\end{kicode}

\begin{kicode}{TutorialD}{}{TutorialD}
table_1 AND table_2;
\end{kicode}

%% Závěry práce. V jazyce práce a anglicky. Text pro jiný než
%% nastavený jazyk práce (nepovinným parametrem language makra
%% \documentclass, výchozí český) se zadává použitím makra s uvedením
%% jazyka jako nepovinného parametru.
\begin{kiconclusions}
Závěr práce v \uv{českém} jazyce.
\end{kiconclusions}

\begin{kiconclusions}[english]
Thesis conclusions in \uv{English}.
\end{kiconclusions}

%% Přílohy obsahu textu práce, za makrem \appendix.
\appendix

\section{První příloha}
Text první přílohy

\section{Druhá příloha}
Text druhé přílohy

%% Obsah přiloženého datového média. Poslední příloha. Upravte podle vlastní
%% práce!
\section{Obsah přiloženého datového média} \label{sec:ObsahMedia}

Na samotném konci textu práce je uveden stručný popis obsahu
přiloženého datového média (CD/DVD, flash disk apod.), tj.~jeho závazné adresářové struktury, důležitých
souborů apod.

\begin{description}

\item[\texttt{bin/}] \hfill \\
  Instalátor \textsc{Instalator} programu, popř.~program
  \textsc{Program}, spustitelné přímo z~média. / Kompletní adresářová
  struktura webové aplikace \textsc{Webovka} (v~ZIP archivu) pro
  zkopírování na webový server. Adresář obsahuje i~všechny runtime
  knihovny a~další soubory potřebné pro bezproblémový běh instalátoru
  a~programu z~média / pro bezproblémový provoz webové aplikace na
  webovém serveru.

\item[\texttt{doc/}] \hfill \\
  Text práce ve formátu PDF, vytvořený s~použitím závazného stylu KI
  PřF UP v~Olomouci pro závěrečné práce, včetně všech příloh,
  a~všechny soubory potřebné pro bezproblémové vygenerování PDF
  dokumentu textu (v~ZIP archivu), tj.~zdrojový text textu, vložené
  obrázky, apod.

\item[\texttt{src/}] \hfill \\
  Kompletní zdrojové texty programu \textsc{Program} / webové aplikace
  \textsc{Webovka} se všemi potřebnými (příp.~převzatými) zdrojovými
  texty, knihovnami a~dalšími soubory potřebnými pro bezproblémové
  vytvoření spustitelných verzí programu / adresářové struktury pro
  zkopírování na webový server.

\item[\texttt{readme.txt}] \hfill \\
  Instrukce pro instalaci a~spuštění programu \textsc{Program}, včetně
  všech požadavků pro jeho bezproblémový provoz. / Instrukce pro
  nasazení webové aplikace \textsc{Webovka} na webový server, včetně
  všech požadavků pro její bezproblémový provoz, a~webová adresa, na
  které je aplikace nasazena pro účel testování při tvorbě posudků
  práce a~pro účel obhajoby práce.

\end{description}

Navíc médium obsahuje:

\begin{description}

\item[\texttt{data/}] \hfill \\
  Ukázková a~testovací data použitá v~práci a~pro potřeby testování
  práce při tvorbě posudků a~obhajoby práce.

\item[\texttt{install/}] \hfill \\
  Instalátory aplikací, runtime knihoven a~jiných souborů potřebných
  pro provoz programu \textsc{Program} / webové aplikace
  \textsc{Webovka}, které nejsou standardní součástí operačního
  systému určeného pro běh programu / provoz webové aplikace.

\item[\texttt{literature/}] \hfill \\
  Vybrané položky bibliografie, příp.~jiná užitečná literatura
  vztahující se k~práci.

\end{description}

U~veškerých cizích převzatých materiálů obsažených na médiu jejich
zahrnutí dovolují podmínky pro jejich šíření nebo přiložený souhlas
držitele copyrightu. Pro všechny použité (a~citované) materiály,
u~kterých toto není splněno a~nejsou tak obsaženy na médiu, je uveden
jejich zdroj (např.~webová adresa) v~bibliografii nebo textu práce
nebo v souboru \texttt{readme.txt}.

%% -------------------------------------------------------------------

%% Sazba volitelného seznamu zkratek, za přílohami.
\printglossary

%% Sazba povinné bibliografie, za přílohami (případně i za seznamem
%% zkratek). Při použití BibLaTeXu použijte makro
%% \printbibliography. jinak prostředí thebibliography. Ne obojí!

%% Sazba i v textu necitovaných zdrojů, při použití
%% BibLaTeXu. Volitelné.
\nocite{*}
%% Vlastní sazba bibliografie při použití BibLaTeXu.
\printbibliography

%% Bibliografie, včetně sazby, při nepoužití BibLaTeXu.
% \begin{thebibliography}{9}
%\bibitem{kniha2} \uppercase{Hawke}, Paul. NanoHttpd: Light-weight HTTP server designed for embedding in other applications. GitHub [online]. 2014-05-12. [cit. 2014-12-06]. Dostupné z: \url{https://github.com/NanoHttpd/nanohttpd}
%
%\bibitem{jeske13} \uppercase{Jeske}, David; \uppercase{Novák}, Josef. Simple HTTP Server in \csharp: Threaded synchronous HTTP Server abstract class, to respond to HTTP requests. CodeProject: For those who code [online]. 2014-05-24. [cit. 2014-12-06]. Dostupné z: \url{http://www.codeproject.com/Articles/137979/Simple-HTTP-Server-in-C}
%
%\bibitem{uzis2012} \uppercase{ÚSTAV ZDRAVOTNICKÝCH INFORMACÍ A STATISTIKY ČR}. Lékaři, zubní lékaři a farmaceuti 2012 [online]. Praha 2, Palackého náměstí 4: Ústav zdravotnických informací a statistiky ČR, 2012 [cit. 2014-12-06]. ISBN 978-80-7472-089-5. Dostupné z: \url{http://www.uzis.cz/publikace/lekari-zubni-lekari-farmaceuti-2012}
% \end{thebibliography}

%% Sazba volitelného rejstříku, za bibliografií.
\printindex

\end{document}

%%% Local Variables:
%%% mode: latex
%%% TeX-master: t
%%% End:
